\chapter{Markov Chain Monte Carlo Methods}

\textit{Markov Chain Monte Carlo}(MCMC) is a powerful collection of algorithms that enable us to simulate from complicated distributions using Markov chains. 
When the target distributions is an unconventional one, or it is known only up to a normalizing constant that is $ f(x) = \frac{h(x)}{c} $ for some explicit function $ h $ but only an implicit normalizing constant $ c $, because $ c $ can not be computed exactly, the standard simulation techniques are difficult to apply or even not applicable in that case \textit{Markov Chain Monte Carlo}(MCMC) comes in to play.
The basic idea for MCMC is to construct a \textit{Markov Chain} whose stationary distribution is the distribution of interest.

MCMC is widely used algorithms it is primarily used for calculating numerical approximations  of multi-dimensional integration for example is Bayesian statistic, computational physics, computational biology.
In Bayesian statistic, Markov Chain Monte Carlo method are typically used to calculate moments and posterior distribution.

To understand \textit{Markov Chain Monte Carlo}(MCMC) we have to understand about \textit{Markov Chains}.  

\section{Markov Chains}

\textbf{Fill if later} 

\chapter{Generating Random Variables}

\section{Generating Discrete Random Variables}

Main component of a simulation study is the ability to generate random number, where a random number represents the value of random variable uniform
distribution on $(0,1)$. 
\subsection{Pseudorandom Number Generation}
Random numbers were originally either manually or mechanically generated, by using spinning wheels or dice rolling or card shuffling 
but the modern approach is to use a computer to successively generate pseudorandom numbers.

One of the common approaches to generate pseudorandom numbers starts with an initial value $x_0$, called seed, and then recursively computes 
successive values $x_n, n\ge1$, by letting
\begin{equation}
    \label{MCM}
    x_n = a x_{n-1} \text{ modulo } m 
\end{equation}
where $a$ and $m$ are given positive integers, and where the equation (\ref{MCM}) means that $ax_{n-1}$ is divided by  $m$ and remainder is taken as the 
value of $x_n$. Thus, each value of $x_n$ is either $0,1, \ldots, m-1$ and the quantity $x_n / m$ is pseudorandom number and follows 
an approximation to the value of a uniform $(0,1)$ random variable.

The approach specified by equation (\ref{MCM}) to generate random numbers is called the Multiplicative Congruential Method.

Another method is 
\[
    x_n = (a x_{n-1}+c) \text{ modulo } m
\] 
this method is known as Mixed Congruential Generators where $c$ is a non-negative integer.

\subsection{The Inverse Transform Method}
Suppose we want to generate the value of a discrete random variable $X$ having probability mass function
 \[
     P(X=x_i)=p_i, \ i = 0,1, \ldots , \ \sum_ip_i =1
\] 
To do this, we generate a random number from a uniform distribution $(0,1)$ $U$, and set
 \[
X=
\begin{cases}
    x_0 \text{ if } U<p_0 \\
    x_1 \text{ if } p_0\le U\le p_0+p_1 \\ 
    \vdots \\
    x_j \text{ if } \sum_{i=0}^{j-1}p_i\le U\le \sum_{i=0}^{j}p_i\\ 
    \vdots
\end{cases}
\] 
Since, for $0<a<b<1, P(a\le U<b) = b-a$, we have,
\[
    P(X=x_j)=P\left( \sum_{i=0}^{j-1}p_i\le U< \sum_{i=0}^{j}p_i \right) = p_j
.\] 
So, $X$ has the desired distribution.

\begin{example}[Bernoulli Distribution]
    Let, $X\sim Ber(p)$ where p is success probability  i.e.  $P(X=0)= 1-p$ and  $P(X=1)=p$ and $0\le p \le 1$.
    Then, to generate $X$ we first generate $U \sim U[0,1]$ then, we set

    \[
        X=
        \begin{cases}
            1, \text{ if } U\le p \\
            0, \text{ if } U> p
        \end{cases}
    \] 
    Hence, $X$ follows Bernoulli Distribution with the parameter $p$.\\ 
    \textbf{Algorithm for Inverse Transform Algorithm for Generating Bernoulli Distribution:}\\ 
    STEP 1: Generate a random variable $U\sim U[0,1]$.\\ 
    STEP 2: If $U\le p$ set $X=1$ or set  $X=0$. \\ 
    STEP 3: Go to  STEP 1.

    \begin{figure}[H]
        \centering
        \includegraphics[width=0.6\textwidth]{images/ber_ITA.png}
        \caption{Inverse Transform plot for generating  Bernoulli random numbers with $p=0.5$}
    \end{figure}
\end{example}

\begin{example}[Binomial Distribution]
   Let, $X\sim Bin(n,p)$ then,  $X$ has probability mass function
   \[
       f(r) = P(X=r) = {n\choose r}p^{r}(1-p)^{n-r},\  i = 1,2, \ldots 
   \] 
   The generation of $X\sim Bin(n,p)$ by Inverse Transform Algorithm can be tedious. We can use the relation between Binomial and Bernoulli distribution.
   If $x_i \sim Ber(p), \forall i = 1,2, \ldots, n$ then, $\sum_{i=1}^{n} x_i\sim Bin(n,p)$.

   Hence, by generating $x_i$ $n$ independent random variable from Bernoulli distribution and summing them we get binomial distribution
   \begin{figure}[H]
       \centering
       \includegraphics[width=0.7\textwidth]{images/bin_ITA.png}
       \caption{Generation of  binomial random numbers with $n=5$ and  $p=0.5$}
   \end{figure}
\end{example}

\section{Generating Continuous Random Variables}

\subsection{The Inverse Transform Algorithm}
To generate Continuous random variables The Inverse Transform Algorithm is very important method. It is based on a following theorem.
\begin{theorem}
    \label{ITA theorem}
    Let $U$ be a uniform  $(0,1)$ random variable. For any continuous distribution function  $F$ the random variable  $X$ defined by 
    \[
    X=F^{-1}(U)
    \] 
    has distribution $F$.
\end{theorem}
\begin{proof}
    Let, $F_X$ denote the distribution function of  $X=F^{-1}(U)$. Then,
    \begin{align*}
        F_X(x) &= P(X\le x)\\ 
               &= P(F^{-1}(U)\le x) \\ 
    \end{align*}
    Since, $F$ is a cumulative distribution function it follows that $F(x)$ is monotonic increasing function of  $x$ and range of  $F(x)$ is  $(0,1)$.
    Then,
    \begin{align*}
        F_X(x) &= P\left(F\left(F^{-1}(U)\right)\le F(x)\right) \\ 
               &= P(U\le F(x))\\ 
               &= F(x) \text{ since $U\sim U(0,1)$ }
    \end{align*}
\end{proof}

The above theory tells us we can generate a random variable $X$ from the continuous distribution function  $F$ by generating a random number $U\sim U(0,1)$ 
and setting $X=F^{-1}(U)$.

\begin{example}[Exponentian Distribution]
    Suppose we want to generate a random variable $x\sim Exp(\lambda)$, then its probability density function is 
    \[
    f(x) = \lambda e^{-\lambda x}.
    \] 
    Hence, The cumulative distribution function is,
    \[
    F(x) = 1-e^{\lambda x}
    \] 
    if we let $x=F^{-1}(u)$, then,
    \begin{align*}
        u&=F(x)=1-e^{-\lambda x} \\ 
         1-u &= e ^{-\lambda x}\\ 
         x &= - \frac{\ln(1-u)}{\lambda}
    \end{align*}
    
    Hence, we can generate an exponential random variable with parameter 1 by generating a uniform $(0,1)$ random number $U$ and then setting
    \[
    X = F^{-1}(U) = -\frac{\ln(1-U)}{\lambda}.
    \] 
    We see that if $U\sim U(0,1)$ then also $1-U\sim U(0,1)$ thus  $\ln(1-U)$ has the same distribution as  $\ln(U)$ so,
    \[
    X = F^{-1}(U) = -\frac{\ln(U)}{\lambda}.
    \] 
    will also work. If we use second expression then the algorithm will take less computing power hence less time.
    \begin{figure}[H]
        
        \centering
        \includegraphics[width=0.7\textwidth]{images/exp_ITA.png}
        \caption{Inverse Transform plot for generating $Exp(2)$}
    \end{figure}
\end{example}
\begin{example}[Gamma Distribution]
    Let $X\sim G(n,\lambda)$ Then, its probability  mass function is given by,
    \[
        f(x) = \frac{1}{\Gamma(n)}\lambda^{n}x^{n-1}e^{-\lambda x}
    \] 

    We know if $X_i\sim Exp(\lambda) \forall i = 1,2, \ldots ,n$ then  $\sum_i X_i \sim G(n,\lambda)$. Then, Generating  $n$ $X_i\sim Exp(\lambda)$ 
    and summing them we can easily generate a random variable which follows gamma distribution
    \begin{figure}[H]
        
        \centering
        \includegraphics[width=0.7\textwidth]{images/gamma_ITA.png}
        \caption{$G(10,5)$ generation by sum of $Exp(5)$}
    \end{figure}
\end{example}

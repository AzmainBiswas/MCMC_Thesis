\chapter{Introduction}

In the ever-evolving landscape of mathematical and statistical research and application, 
the integration of simulation techniques has emerged as a powerful tool to unravel complex phenomena, validate theoretical frameworks, 
and facilitate a deeper understanding of intricate mathematical structures. 
Simulation is a computer-based exploratory exercise that aids in understanding how
the behavior of a random or even a deterministic process changes in response to
changes in input or the environment. It is essentially the only option left when exact
mathematical calculations are impossible, or require an amount of effort that the user
is not willing to invest. Even when the mathematical calculations are quite doable, a
preliminary simulation can be very helpful in guiding the researcher to theorems that
were not a priori obvious or conjectured, and also to identify the more productive
corners of a particular problem. Although simulation in itself is a machine-based
exercise, credible simulation must be based on appropriate theory. A simulation
algorithm must be theoretically justified before we use it.

The classic theory of simulation includes such time-tested methods as the original Monte Carlo, 
Inverse Transform method, Accept-Reject method, Bivariate techniques  
from standard distributions in common use. They involve a varied degree of sophistication. 
Markov chain Monte Carlo is the name for a collection of simulation algorithms for simulating from
the distribution of very general types of random variables taking values in quite
general spaces. MCMC methods have truly revolutionized simulation because of an
inherent simplicity in applying them, the generality of their scopes, and the diversity
of applied problems in which some suitable form of MCMC has helped in making
useful practical advances. MCMC methods are the most useful when conventional
Monte Carlo is difficult or impossible to use.

Simulation depend on various theoretical aspect such as 
The weak law of Large Number, The Central limit theory, The sample mean and sample variance etcetera.

There are various type of simulation technique in standard simulation theory 
such as,

\begin{enumerate}
	\itemsep=-.3em
	\item The Inverse Transform Method
	\item Accept-Reject Algorithm
	\item Bivariate Techniques
	\item Ordinary Monte Carlo
	\item Importance Sampling
	\item Markov Chain Monte Carlo
\end{enumerate}

This project work focus mainly on these simulation techniques.

In Chapter 2, we discuss about how to generate random variable both
uniform and continuous, by using method like The Inverse Transform Method, 
Accept-Reject Algorithm, Bivariate Techniques.

In Chapter 3, we focus on Ordinary Monte Carlo and how to use it to solve problem like evaluating integration and evaluating the value of $\pi$.

In Chapter 4, the focus shifts to Importance Sampling and how it beneficial
from Ordinary Monte Carlo by some example. Learn about how to chose 
optimal Importance sample distribution.

\section{Mathematical Preliminaries}

\begin{definition}[Probability Space]
	A probability spacs is a triple $(\Omega, \mathcal{F}, P)$ consisting of:
	\begin{enumerate}
		\item[(a)] the sample space $\Omega$ (an arbitrary non-empty set)
		\item[(b)] a non-empty collection of subsets $\mathcal{F} $ of $ \Omega $,
		      called \textit{sigma field} of subspace of $ \Omega $, 
		      such taht,
		      \begin{itemize}
			      \item[(i)] $ \Omega \in \mathcal{F} $
			      \item[(ii)] if $ A \in \mathcal{F} $, then $ A^{c}\in \mathcal{F}  $
			      \item [(iii)] if $ A_n\in \mathcal{F},\ n=1,2,\ldots $, then $ \cup_n=1^{\infty}A_n \in \mathcal{F}  $
		      \end{itemize}
		\item [(c)] a probability measure $ p:\mathcal{F}\to [0,1] $, which is a real valued function on $ \mathcal{F} $
		      such that,
		      \begin{itemize}
			      \item [(i)] $ P(A) \ge 0 $ for all $ A\in \mathcal{F} $
			      \item [(ii)] $ P(\Omega) = 1 $
			      \item [(iii)] if $ A_1,A_2,\ldots $ are disjoint sets in $ \mathcal{F} $, then $ P(\cup_n A_n) = \sum_{n} P(A_n)  $.
		      \end{itemize}
	\end{enumerate}
\end{definition}

\begin{definition}[Conditional Probability]
	Let $A$, $B$ be general events with respect to some sample space $ \Omega $,
	and suppose $ P(A)>0 $. The \textit{conditional probability} of $ B $ given $ A $ is defined as 
	\[
		P(B|A) = \frac{P(A\cap B}{P(A)}.
	\]
\end{definition}

\begin{theorem}[Bayes's Theorem]
	Let, $ \{ A_1,A_2,\ldots,A_n \} $ be a partition of sample space $ \Omega $. 
	Let $ B $ be a some fixed event. Then
	\[
		P(A_j|B) = \frac{P(B|A_j)P(A_j)}{\sum_{i=1}^{n}P(B|A_i)P(A_i) }.
	\]
\end{theorem}

\begin{definition}[Random Variable]
	Let, $ \Omega $ be a sample space corresponding to some experiment and let 
	$ X:\Omega\to \mathds{R} $ be a function from the sample space to the real line. 
	Then $ X $ is called a \textit{random Variable}
\end{definition}

\begin{definition}[Cumulative Distribution Function]
	The cumulative distribution function(CDF) or simply distributed function, $ F $ of 
	a random variable $ X $ is define for any real number $ x $ by
	\[
		F(x) = P(X \le x).
	\]
\end{definition}

\begin{definition}[Probability Mass Function]
	For a discrete random variable $ X $ we define its Probability mass function(pmf)
	$ p(x) $ by
	\[
		p(x) = P(X=x)
	\]
	and we have,
	\[
		\sum_{i\in \Lambda} p(x_i) = 1. \text{ and } p_i \ge 0. 
	\]
\end{definition}

\begin{definition}[ Probability Density Function ]
	For a continuous random variable if there is a non-negative function $ f(x) $
	defined for all real number $ x $ and having the property that for any set $ C\subset \mathds{R} $,
	\[
		P(X\in C) = \int_{C}f(x) dx 
		.\]
	The function $ f $ is called probability density function(pdf) of the random variable $ X $.
\end{definition}

The relation between CDF and pdf is express by,
\[
	F(a) = P(X\in (-\infty, a]) = \int_{-\infty}^{a} f(x) dx.
\]

\begin{definition}[Exception]
	If $ X $ is a random variable,
	then the \textit{exception} or \textit{expected value} of $ X $, also called the mean of $ X $ and denoted by $ E(X) $, is define by
	\[
		E(X) = \int xdF(x) 
	\]
\end{definition}

\begin{definition}[Variance]
	If $ X $ is a random variable with mean $ E(X) $, then the variance of $ X $,
	denoted by $ Var(X) $, is define by,
	\[
		Var(X)=E\left( (X-E(X))^{2} \right)
	\]
\end{definition}

\begin{theorem}[The Weak Law of Large Numbers]
	Let $X_1,X_2,\ldots$ be a sequence of in dependent and identically distributed 
	random variables having mean $ \mu $, Then, for any $ \epsilon >0 $,
	\[
		P \left( \left|\frac{X_1+X_2+ \ldots + X_n }{n} - \mu \right| > \epsilon \right) \to 0 
	\]
\end{theorem}

\begin{theorem}[The Central Limit Theorem]
	Suppose $X_1, X_2, \ldots  $ is a sequence of i.i.d random variables with 
	$E[X_i]=\mu$ and $Var[X_i]=\sigma ^{2} < \infty$. Then, 
	\[
		\lim_{n \to \infty} P\left( \frac{X_1+\ldots+X_n - n\mu}{\sigma \sqrt{n } } < n \right) = \Phi(Z)
	\]
	Were, $\Phi(Z)$ denote the distribution function of a standard normal distribution.
\end{theorem}

\subsection*{Sample Mean and Sample Variance}
Suppose that $ X_1,X_2,\ldots,X_n $ are independent random variable having the same
distribution function. Let $ \mu $ and $ \sigma ^{2} $ denote, respectively 
their mean and variance that is, $ \mu = E(X_i) $ and $ \sigma ^{2} = Var(X_i) $.
The quantity
\[
	\bar{X} \equiv \sum_{i=1}^{n} \frac{X_i}{n}
\]

which is the arithmetic average of the $ n $ data values, is called the \textit{sample mean}.
When the population mean $ \mu $ is unknown, the sample mean is often used to estimate $ \mu .$

Because, 
\begin{align*}
	E(\bar{X}) & = E\left( \sum_{i=1}^{n} \frac{X_i}{n} \right) \\ 
	           & = \sum_{i=1}^{n} \frac{E(X_i)}{n}              \\ 
	           & = \frac{n \mu}{n} = \mu
\end{align*}

if follows that $ \bar{X} $ is an unbiased estimator of $ \mu $, where we say that an estimator
of parameter is an unbiased estimator of that an estimator of a parameter is an unbiased estimator
of that parameter if its expected value is equal to the parameter.

The quantity $ S ^{2} $, define by 
\[
	S ^{2} = \frac{\sum_{i=1}^{n}(X_i - \bar(X))^{2} }{n-1}
\]
is called the \textit{sample variance} .

Whish is also unbised of $ \sigma ^{2} $

\chapter{Introduction}

In the ever-evolving landscape of mathematical and statistical research and application, 
the integration of simulation techniques has emerged as a powerful tool to unravel complex phenomena, validate theoretical frameworks, 
and facilitate a deeper understanding of intricate mathematical structures. 
Simulation is a computer-based exploratory exercise that aids in understanding how
the behavior of a random or even a deterministic process changes in response to
changes in input or the environment. It is essentially the only option left when exact
mathematical calculations are impossible, or require an amount of effort that the user
is not willing to invest. Even when the mathematical calculations are quite doable, a
preliminary simulation can be very helpful in guiding the researcher to theorems that
were not a priori obvious or conjectured, and also to identify the more productive
corners of a particular problem. Although simulation in itself is a machine-based
exercise, credible simulation must be based on appropriate theory. A simulation
algorithm must be theoretically justified before we use it.

The classic theory of simulation includes such time-tested methods as the original Monte Carlo, 
Inverse Transform method, Accept-Reject method, Bivariate techniques  
from standard distributions in common use. They involve a varied degree of sophistication. 
Markov chain Monte Carlo is the name for a collection of simulation algorithms for simulating from
the distribution of very general types of random variables taking values in quite
general spaces. MCMC methods have truly revolutionized simulation because of an
inherent simplicity in applying them, the generality of their scopes, and the diversity
of applied problems in which some suitable form of MCMC has helped in making
useful practical advances. MCMC methods are the most useful when conventional
Monte Carlo is difficult or impossible to use.

Simulation depend on various theoretical aspect such as 
The weak law of Large Number, The Central limit theory, The sample mean and sample variance etcetera.

\section{Mathematical Preliminaries}

\begin{theorem}[The Weak Law of Large Numbers]
    Let $X_1,X_2,\ldots$ be a sequence of in dependent and identically distributed 
    random variables having mean $ \mu $, Then, for any $ \epsilon >0 $,
    \[
        P \left( \left|\frac{X_1+X_2+ \ldots + X_n }{n} - \mu \right| > \epsilon \right) \to 0 
    \]
\end{theorem}

\begin{theorem}[The Central Limit Theorem]
    Suppose $X_1, X_2, \ldots  $ is a sequence of i.i.d random variables with 
    $E[X_i]=\mu$ and $Var[X_i]=\sigma ^{2} < \infty$. Then, 
    \[
        \lim_{n \to \infty} P\left( \frac{X_1+\ldots+X_n - n\mu}{\sigma \sqrt{n } } < n \right) = \Phi(Z)
    \]
    Were, $\Phi(Z)$ denote the distribution function of a standard normal distribution.
\end{theorem}


